%% Generated by Sphinx.
\def\sphinxdocclass{report}
\documentclass[a4paper,10pt,oneside,french]{sphinxmanual}
\ifdefined\pdfpxdimen
   \let\sphinxpxdimen\pdfpxdimen\else\newdimen\sphinxpxdimen
\fi \sphinxpxdimen=.75bp\relax
\ifdefined\pdfimageresolution
    \pdfimageresolution= \numexpr \dimexpr1in\relax/\sphinxpxdimen\relax
\fi
%% let collapsible pdf bookmarks panel have high depth per default
\PassOptionsToPackage{bookmarksdepth=5}{hyperref}

\PassOptionsToPackage{warn}{textcomp}
\usepackage[utf8]{inputenc}
\ifdefined\DeclareUnicodeCharacter
% support both utf8 and utf8x syntaxes
  \ifdefined\DeclareUnicodeCharacterAsOptional
    \def\sphinxDUC#1{\DeclareUnicodeCharacter{"#1}}
  \else
    \let\sphinxDUC\DeclareUnicodeCharacter
  \fi
  \sphinxDUC{00A0}{\nobreakspace}
  \sphinxDUC{2500}{\sphinxunichar{2500}}
  \sphinxDUC{2502}{\sphinxunichar{2502}}
  \sphinxDUC{2514}{\sphinxunichar{2514}}
  \sphinxDUC{251C}{\sphinxunichar{251C}}
  \sphinxDUC{2572}{\textbackslash}
\fi
\usepackage{cmap}
\usepackage[T1]{fontenc}
\usepackage{amsmath,amssymb,amstext}
\usepackage{babel}



\usepackage{tgtermes}
\usepackage{tgheros}
\renewcommand{\ttdefault}{txtt}



\usepackage[Sonny]{fncychap}
\ChNameVar{\Large\normalfont\sffamily}
\ChTitleVar{\Large\normalfont\sffamily}
\usepackage{sphinx}

\fvset{fontsize=auto}
\usepackage{geometry}


% Include hyperref last.
\usepackage{hyperref}
% Fix anchor placement for figures with captions.
\usepackage{hypcap}% it must be loaded after hyperref.
% Set up styles of URL: it should be placed after hyperref.
\urlstyle{same}


\usepackage{sphinxmessages}
\setcounter{tocdepth}{3}
\setcounter{secnumdepth}{3}


\title{Diacamma Pro}
\date{mars 11, 2022}
\release{2.6.0}
\author{sd-libre}
\newcommand{\sphinxlogo}{\sphinxincludegraphics{DiacammaPro.jpg}\par}
\renewcommand{\releasename}{Version}
\makeindex
\begin{document}

\ifdefined\shorthandoff
  \ifnum\catcode`\=\string=\active\shorthandoff{=}\fi
  \ifnum\catcode`\"=\active\shorthandoff{"}\fi
\fi

\pagestyle{empty}
\sphinxmaketitle
\pagestyle{plain}
\sphinxtableofcontents
\pagestyle{normal}
\phantomsection\label{\detokenize{index::doc}}



\chapter{Diacamma Pro}
\label{\detokenize{pro/index:diacamma-pro}}\label{\detokenize{pro/index::doc}}
\sphinxAtStartPar
Présentation du logiciel Diacamma Pro.


\section{Présentation}
\label{\detokenize{pro/presentation:presentation}}\label{\detokenize{pro/presentation::doc}}

\subsection{Description}
\label{\detokenize{pro/presentation:description}}
\sphinxAtStartPar
\sphinxstyleemphasis{Diacamma Pro} est un logiciel de gestion spécialement conçu pour les TPE et les micro\sphinxhyphen{}entreprise.

\sphinxAtStartPar
L’application de base est entièrement gratuite et vous permet de gérer les accès à vos données et au carnet d’adresses de votre association. Les modules complémentaires vous permettront d’adapter en quelques clics le logiciel à vos besoins.

\sphinxAtStartPar
Les différents modules disponibles vous permettront, par exemple, de:
\begin{itemize}
\item {} 
\sphinxAtStartPar
Gérer vos documents de façon centralisée grâce à la gestion documentaire.

\item {} 
\sphinxAtStartPar
Gérer la comptabilité de votre association.

\item {} 
\sphinxAtStartPar
Gérer vos devis, factures et factures proformat pour les adhérents subventionnés par un CE, entre autre.

\end{itemize}

\sphinxAtStartPar
Ce manuel vous aidera dans l’utilisation de ce logiciel.
Si malgré tout, vous ne trouvez pas la réponse à vos problèmes, visiter notre site \sphinxurl{https://www.diacamma.org} où vous trouverez des tutoriels et des astuces.


\subsection{Installation}
\label{\detokenize{pro/presentation:installation}}
\sphinxAtStartPar
Vous pouvez installer \sphinxstyleemphasis{Diacamma Pro} sur un ordinateur dédié à votre association que ce soit un Apple Macintosh (macOS 10.12 « Sierra » et +) ou bien un PC sous MS\sphinxhyphen{}Windows (8 et +) ou sous GNU Linux (Ubuntu 16.04 ou +)..

\sphinxAtStartPar
\sphinxstyleemphasis{Diacamma Pro} est un logiciel client/serveur : vous pouvez l’installer sur un ordinateur centralisateur et accéder aux données depuis un autre PC connecté au premier, sans limite du nombre d’utilisateurs simultanés.
Si le PC contenant les données est connecté de manière permanente à internet, vous aurez accès à vos données depuis n’importe où dans le monde!

\sphinxAtStartPar
Cette organisation est particulièrement intéressante pour permette à plusieurs cadres associatifs d’avoir accès à des données communes.

\sphinxAtStartPar
Quel responsable ne s’est pas arraché les cheveux suite à un échanges de documents via une clef USB où certaines modifications importantes se perdent?

\sphinxAtStartPar
Pour plus d’information, visiter notre site \sphinxurl{https://www.diacamma.org}


\subsection{Aides et support}
\label{\detokenize{pro/presentation:aides-et-support}}
\sphinxAtStartPar
Sur le site officiel du logiciel, \sphinxurl{https://www.diacamma.org}, où vous trouverez des tutoriels et un forum pour échanger des astuces entre utilisateurs.

\sphinxAtStartPar
Si vous souhaitez un service et un support plus personnalisé, vous pouvez faire confiance à notre partenaire officiel SLETO.
Pour en savoir plus sur des solutions d’hébergement et de support, rendez vous sur \sphinxurl{https://www.sleto.net}


\section{Prise en main}
\label{\detokenize{pro/first_step:prise-en-main}}\label{\detokenize{pro/first_step::doc}}
\sphinxAtStartPar
Le logiciel \sphinxstyleemphasis{Diacamma Pro} comprends un grand nombre de paramétrages et peu paraître difficile à configurer au besoin de votre structure.

\sphinxAtStartPar
Nous vous proposons cette explication pour vous aider à franchir cette première étape dans l’utilisation de cet outil.

\sphinxAtStartPar
Suivez pas à pas les différents phases de réglages. Dans chaque étape, nous ne ré\sphinxhyphen{}détaillons pas les fonctionnalités.
Nous vous invitons également à vous référer au reste du manuel utilisateur pour cela.

\sphinxAtStartPar
Il peut être intéressant de réaliser des sauvegardes au cours de cette procédure.
Cela vous permettra, si vous faite une erreur, de revenir à une étape précédente sans tout recommencer (installation comprise).


\subsection{Vérifiez la mise à jours de votre logiciel}
\label{\detokenize{pro/first_step:verifiez-la-mise-a-jours-de-votre-logiciel}}
\sphinxAtStartPar
Commencez par vérifiez que votre logiciel est à jours.
En effet, nous diffusons régulièrement des correctifs qui ne sont pas toujours inclus dans les installateurs.


\subsection{Présentation de votre structures}
\label{\detokenize{pro/first_step:presentation-de-votre-structures}}\begin{quote}

\sphinxAtStartPar
Menu \sphinxstyleemphasis{Général/Nos coordonnées}
\end{quote}

\sphinxAtStartPar
Dans cette écran, vous pouvez décrire les coordonnées de votre structure.
De nombreuses fonctionnalités utilisent ces informations en particulier pour les impressions (facture, comptabilité, listing d’adhérents…)


\subsection{Réglage de votre facturier}
\label{\detokenize{pro/first_step:reglage-de-votre-facturier}}\begin{quote}
\begin{quote}

\sphinxAtStartPar
Menu \sphinxstyleemphasis{Administration/Modules (conf.)/Configuration du règlement}

\sphinxAtStartPar
Menu \sphinxstyleemphasis{Administration/Modules (conf.)/Configuration commercial du facturier}
\end{quote}

\sphinxAtStartPar
Menu \sphinxstyleemphasis{Administration/Modules (conf.)/Configuration financière du facturier}
\end{quote}

\sphinxAtStartPar
Vérifiez que les paramètres du facturier vous correspondent.
Vous pouvez aussi ici, configurer le RIB de votre compte bancaire principal ainsi qu’en ajouter des secondaires.
Ces derniers vous seront utiles pour la fonctionnalité « dépôt de chèques » du facturier.


\subsection{Création de votre premier exercice comptable}
\label{\detokenize{pro/first_step:creation-de-votre-premier-exercice-comptable}}\begin{quote}

\sphinxAtStartPar
Menu \sphinxstyleemphasis{Administration/Modules (conf.)/Configuration comptables}

\sphinxAtStartPar
Menu \sphinxstyleemphasis{Finance/Comptabilité/plan comptable}
\end{quote}

\sphinxAtStartPar
Ouvrer votre premier exercice comptable et rendez le actifs.
Vous devrez aussi créer le plan comptable de cette exercice pour avoir une comptabilité pleinement opérationnel.


\subsection{Mise à jours comptable}
\label{\detokenize{pro/first_step:mise-a-jours-comptable}}\begin{quote}

\sphinxAtStartPar
Menu \sphinxstyleemphasis{Finance/Comptabilité/écritures comptable}

\sphinxAtStartPar
Menu \sphinxstyleemphasis{Finance/Comptabilité/Modèle d’écriture}
\end{quote}

\sphinxAtStartPar
Si vous mettez en place \sphinxstyleemphasis{Diacamma Asso} au cours de votre exercice, vous devrez également saisir votre report à nouveau de l’exercice précédent ainsi que ressaisir les écritures du début d’année.
Attention: n’oubliez pas que l’ajout d’une cotisation d’adhérent génère une facture ainsi que des écritures comptables associées. Prenez en compte dans la reprise de votre comptabilité.
Pour vous aidez dans la saisie de votre comptabilité, nous vous conseillons d’utiliser les modèles d’écritures. Enregistrez entant que modèle les écritures récurrentes que vous avez au cours d’une année. Ainsi vous pouvez rapidement compléter votre comptabilité en quelques cliques.


\subsection{Le courriel}
\label{\detokenize{pro/first_step:le-courriel}}\begin{quote}

\sphinxAtStartPar
Menu \sphinxstyleemphasis{Administration/Modules (conf.)/Paramètrages de courriel}
\end{quote}

\sphinxAtStartPar
Définissez vos réglages pour votre courriel.
Le serveur smpt permettra à \sphinxstyleemphasis{Diacamma Asso} d’envoyé un certain nombre de message: facture en PDF, mot de passe de connexion, …
Vous pouvez préciser comment réagis les liens “écrire à tous” réagis avec votre logiciel de messagerie.


\subsection{La gestions documentaire}
\label{\detokenize{pro/first_step:la-gestions-documentaire}}\begin{quote}

\sphinxAtStartPar
Menu \sphinxstyleemphasis{Administration/Modules (conf.)/Dossier}

\sphinxAtStartPar
Menu \sphinxstyleemphasis{Bureautique/Documents/Documents}
\end{quote}

\sphinxAtStartPar
Définissez vos différents dossier vous permettant d’importer vos documents à classer et à partager.

\sphinxAtStartPar
Une fois avoir parcouru ces points, votre logiciel \sphinxstyleemphasis{Diacamma Pro} est pleinement opérationnel.
N’hésitez pas à consulter le forum: de nombreuses astuces peux vous aider pour utiliser au mieux votre logiciel.



\renewcommand{\indexname}{Index}
\printindex
\end{document}