%% Generated by Sphinx.
\def\sphinxdocclass{report}
\documentclass[a4paper,10pt,oneside,french]{sphinxmanual}
\ifdefined\pdfpxdimen
   \let\sphinxpxdimen\pdfpxdimen\else\newdimen\sphinxpxdimen
\fi \sphinxpxdimen=.75bp\relax
\ifdefined\pdfimageresolution
    \pdfimageresolution= \numexpr \dimexpr1in\relax/\sphinxpxdimen\relax
\fi
%% let collapsible pdf bookmarks panel have high depth per default
\PassOptionsToPackage{bookmarksdepth=5}{hyperref}


\PassOptionsToPackage{warn}{textcomp}
\usepackage[utf8]{inputenc}
\ifdefined\DeclareUnicodeCharacter
% support both utf8 and utf8x syntaxes
  \ifdefined\DeclareUnicodeCharacterAsOptional
    \def\sphinxDUC#1{\DeclareUnicodeCharacter{"#1}}
  \else
    \let\sphinxDUC\DeclareUnicodeCharacter
  \fi
  \sphinxDUC{00A0}{\nobreakspace}
  \sphinxDUC{2500}{\sphinxunichar{2500}}
  \sphinxDUC{2502}{\sphinxunichar{2502}}
  \sphinxDUC{2514}{\sphinxunichar{2514}}
  \sphinxDUC{251C}{\sphinxunichar{251C}}
  \sphinxDUC{2572}{\textbackslash}
\fi
\usepackage{cmap}
\usepackage[T1]{fontenc}
\usepackage{amsmath,amssymb,amstext}
\usepackage{babel}



\usepackage{tgtermes}
\usepackage{tgheros}
\renewcommand{\ttdefault}{txtt}



\usepackage[Sonny]{fncychap}
\ChNameVar{\Large\normalfont\sffamily}
\ChTitleVar{\Large\normalfont\sffamily}
\usepackage{sphinx}

\fvset{fontsize=auto}
\usepackage{geometry}


% Include hyperref last.
\usepackage{hyperref}
% Fix anchor placement for figures with captions.
\usepackage{hypcap}% it must be loaded after hyperref.
% Set up styles of URL: it should be placed after hyperref.
\urlstyle{same}


\usepackage{sphinxmessages}
\setcounter{tocdepth}{3}
\setcounter{secnumdepth}{3}


\title{Diacamma Asso}
\date{oct. 28, 2022}
\release{2.6.6}
\author{sd-libre}
\newcommand{\sphinxlogo}{\sphinxincludegraphics{DiacammaAsso.jpg}\par}
\renewcommand{\releasename}{Version}
\makeindex
\begin{document}

\ifdefined\shorthandoff
  \ifnum\catcode`\=\string=\active\shorthandoff{=}\fi
  \ifnum\catcode`\"=\active\shorthandoff{"}\fi
\fi

\pagestyle{empty}
\sphinxmaketitle
\pagestyle{plain}
\sphinxtableofcontents
\pagestyle{normal}
\phantomsection\label{\detokenize{index::doc}}


\sphinxstepscope


\chapter{Diacamma Asso}
\label{\detokenize{asso/index:diacamma-asso}}\label{\detokenize{asso/index::doc}}
\sphinxAtStartPar
Présentation du logiciel Diacamma Asso.

\sphinxstepscope


\section{Présentation}
\label{\detokenize{asso/presentation:presentation}}\label{\detokenize{asso/presentation::doc}}

\subsection{Description}
\label{\detokenize{asso/presentation:description}}
\sphinxAtStartPar
\sphinxstyleemphasis{Diacamma Asso} est un logiciel de gestion spécialement conçu pour les associations sportives ou culturelles.
Avec \sphinxstyleemphasis{Diacamma Asso}, donnez à votre association le logiciel qu’elle mérite. Pas besoin d’être informaticien pour avoir les outils adaptés à votre cas.

\sphinxAtStartPar
L’application de base est entièrement gratuite et vous permet de gérer les accès à vos données et au carnet d’adresses de votre association.

\sphinxAtStartPar
Les différents modules disponibles vous permettront, par exemple, de :
\begin{itemize}
\item {} 
\sphinxAtStartPar
Gérer plus spécifiquement vos adhérents avec leurs licences sportives et le suivi de leur parcours.

\item {} 
\sphinxAtStartPar
Gérer vos documents de façon centralisée grâce à la gestion documentaire.

\item {} 
\sphinxAtStartPar
Gérer la comptabilité de votre association.

\item {} 
\sphinxAtStartPar
Gérer vos devis, factures et factures proformat pour les adhérents subventionnés par un CE, entre autre.

\end{itemize}

\sphinxAtStartPar
Ce manuel vous aidera dans l’utilisation de ce logiciel.
Si malgré tout, vous ne trouvez pas la réponse à vos problèmes, visiter notre site \sphinxurl{https://www.diacamma.org} où vous trouverez des tutoriels et des astuces.


\subsection{Installation}
\label{\detokenize{asso/presentation:installation}}
\sphinxAtStartPar
Vous pouvez installer \sphinxstyleemphasis{Diacamma Asso} sur un ordinateur dédié à votre association que ce soit un Apple Macintosh (macOS 10.12 « Sierra » et +) ou bien un PC sous MS\sphinxhyphen{}Windows (8 et +) ou sous GNU Linux (Ubuntu 16.04 ou +).

\sphinxAtStartPar
\sphinxstyleemphasis{Diacamma Asso} est un logiciel client/serveur : vous pouvez l’installer sur un ordinateur centralisateur et accéder aux données depuis un autre PC connecté au premier, sans limite du nombre d’utilisateurs simultanés.
Si le PC contenant les données est connecté de manière permanente à internet, vous aurez accès à vos données depuis n’importe où dans le monde!

\sphinxAtStartPar
Cette organisation est particulièrement intéressante pour permettre à plusieurs cadres associatifs d’avoir accès à des données communes.

\sphinxAtStartPar
Quel responsable ne s’est pas arraché les cheveux suite à un échange de documents via une clef USB où certaines modifications importantes se perdent?

\sphinxAtStartPar
Pour plus d’information, visiter notre site \sphinxurl{https://www.diacamma.org}


\subsection{Aides et support}
\label{\detokenize{asso/presentation:aides-et-support}}
\sphinxAtStartPar
Sur le site officiel du logiciel, \sphinxurl{https://www.diacamma.org}, où vous trouverez des tutoriels et un forum pour échanger des astuces entre utilisateurs.

\sphinxAtStartPar
Si vous souhaitez un service et un support plus personnalisé, vous pouvez faire confiance à notre partenaire officiel SLETO.
Pour en savoir plus sur des solutions d’hébergement et de support, rendez vous sur \sphinxurl{https://www.sleto.net}

\sphinxstepscope


\section{Prise en main}
\label{\detokenize{asso/first_step:prise-en-main}}\label{\detokenize{asso/first_step::doc}}
\sphinxAtStartPar
Le logiciel \sphinxstyleemphasis{Diacamma Asso} comprend un grand nombre de paramétrages et peut paraître difficile à configurer aux besoins de votre structure.

\sphinxAtStartPar
Nous vous proposons cette explication pour vous aider à franchir cette première étape dans l’utilisation de cet outil.
Un assistant de configuration reprend ces différentes étapes pour vous aider également à configurer ce logiciel.

\sphinxAtStartPar
Suivez pas à pas les différentes phases de réglages. Dans chaque étape, nous ne détaillons pas les fonctionnalités.
Nous vous invitons à vous référer au reste du manuel utilisateur pour cela.

\sphinxAtStartPar
Il peut être intéressant de réaliser des sauvegardes au cours de cette procédure.
Cela vous permettra, si vous faite une erreur, de revenir à une étape précédente sans tout recommencer (installation comprise).


\subsection{Vérifiez la mise à jour de votre logiciel}
\label{\detokenize{asso/first_step:verifiez-la-mise-a-jour-de-votre-logiciel}}
\sphinxAtStartPar
Commencez par vérifiez que votre logiciel est à jour.
En effet, nous diffusons régulièrement des correctifs qui ne sont pas toujours inclus dans les installateurs.


\subsection{Présentation de votre structures}
\label{\detokenize{asso/first_step:presentation-de-votre-structures}}\begin{quote}

\sphinxAtStartPar
Menu \sphinxstyleemphasis{Général/Nos coordonnées}
\end{quote}

\sphinxAtStartPar
Dans cette écran, vous pouvez décrire les coordonnées de votre structure.
De nombreuses fonctionnalités utilisent ces informations en particulier pour les impressions (facture, comptabilité, listing d’adhérents…)


\subsection{Gestion des adhérents : la saison et les cotisations}
\label{\detokenize{asso/first_step:gestion-des-adherents-la-saison-et-les-cotisations}}\begin{quote}

\sphinxAtStartPar
Menu \sphinxstyleemphasis{Administration/Modules (conf.)/Les saisons et les cotisations}
\end{quote}

\sphinxAtStartPar
Vérifiez que vous êtes sur la bonne saison active. Vous pouvez également changer les périodes associées à cette saison et lui associer des documents demandés liés à toutes nouvelles cotisations.
En début de saison, vous pouvez modifier la liste des différentes cotisations que vous proposez à vos adhérents.
Si vous utilisez le module de facturation, l’outil vous permettra d’associer à une cotisation des articles et un montant qui sera utilisé automatiquement.


\subsection{Gestion des adhérents : activités et équipes}
\label{\detokenize{asso/first_step:gestion-des-adherents-activites-et-equipes}}\begin{quote}

\sphinxAtStartPar
Menu \sphinxstyleemphasis{Administration/Modules (conf.)/Catégories}
\end{quote}

\sphinxAtStartPar
Dans le cas où vous voulez gérer plusieurs activités, équipes ou trier vos adhérents par classe d’âges, veuillez configurer ces différentes listes.


\subsection{Réglage de votre facturier}
\label{\detokenize{asso/first_step:reglage-de-votre-facturier}}\begin{quote}

\sphinxAtStartPar
Menu \sphinxstyleemphasis{Administration/Modules (conf.)/Configuration du règlement}

\sphinxAtStartPar
Menu \sphinxstyleemphasis{Administration/Modules (conf.)/Configuration commerciale du facturier}

\sphinxAtStartPar
Menu \sphinxstyleemphasis{Administration/Modules (conf.)/Configuration financière du facturier}
\end{quote}

\sphinxAtStartPar
Vérifiez que les paramètres du facturier vous correspondent.
Vous pouvez aussi ici, configurer le RIB de votre compte bancaire principal ainsi qu’en ajouter d’autres comptes courants.
Ces derniers vous seront utiles pour la fonctionnalité « dépôt de chèques » du facturier.


\subsection{Création de votre premier exercice comptable}
\label{\detokenize{asso/first_step:creation-de-votre-premier-exercice-comptable}}\begin{quote}

\sphinxAtStartPar
Menu \sphinxstyleemphasis{Administration/Modules (conf.)/Configuration comptables}

\sphinxAtStartPar
Menu \sphinxstyleemphasis{Comptabilité/Gestion comptable/plan comptable}
\end{quote}

\sphinxAtStartPar
Ouvrer votre premier exercice comptable et rendez le actif.
Vous devrez aussi créer le plan comptable de cet exercice pour avoir une comptabilité pleinement opérationnelle.


\subsection{Ajout des adhérents de l’association}
\label{\detokenize{asso/first_step:ajout-des-adherents-de-l-association}}\begin{quote}

\sphinxAtStartPar
Menu \sphinxstyleemphasis{Association/Adhérents/Adhérents cotisants}

\sphinxAtStartPar
Menu \sphinxstyleemphasis{Administration/Modules (conf.)/Importation de contacts}
\end{quote}

\sphinxAtStartPar
Il vous faut maintenant ajouter les différents adhérents de votre association.
Deux solutions pour cela: un par un ou via une importation.
Pour les ajouter un par un, vous devez ajouter une fiche pour chacun de vos adhérents.
Si vous avez une liste d’adhérents à importer, extrayez\sphinxhyphen{}la au format CSV, en respectant la structure de fichier expliquée dans le manuel.

\sphinxAtStartPar
N’oubliez pas de leur attribuer une cotisation, sinon ils n’apparaîtrons pas dans la liste des adhérents cotisants.
En cas d’erreur, vous pouvez facilement les retrouver à l’aide de l’outil de recherche.

\sphinxAtStartPar
Notez que pour chaque nouvelle cotisation saisie, une facture non\sphinxhyphen{}validée sera générée automatiquement.


\subsection{Mise à jour comptable}
\label{\detokenize{asso/first_step:mise-a-jour-comptable}}\begin{quote}

\sphinxAtStartPar
Menu \sphinxstyleemphasis{Comptabilité/Gestion comptable/écritures comptables}

\sphinxAtStartPar
Menu \sphinxstyleemphasis{Comptabilité/Gestion comptable/Modèles d’écriture}
\end{quote}

\sphinxAtStartPar
Si vous mettez en place \sphinxstyleemphasis{Diacamma Asso} au cours de votre exercice comptable, vous devrez également saisir les à\sphinxhyphen{}nouveaux et le report\sphinxhyphen{}à\sphinxhyphen{}nouveau de l’exercice comptable précédent ainsi que saisir les écritures du début d’année.
Attention: n’oubliez pas que l’ajout d’une cotisation d’adhérent génère une facture ainsi que des écritures comptables associées. Tenez\sphinxhyphen{}en compte dans la reprise de votre comptabilité.
Pour vous aidez dans la saisie de votre comptabilité, nous vous conseillons d’utiliser les modèles d’écritures. Enregistrez en tant que modèles les écritures récurrentes que vous avez au cours d’une année. Ainsi vous pouvez rapidement compléter votre comptabilité en quelques clics.


\subsection{Le courriel}
\label{\detokenize{asso/first_step:le-courriel}}\begin{quote}

\sphinxAtStartPar
Menu \sphinxstyleemphasis{Administration/Modules (conf.)/Paramètrages de courriel}
\end{quote}

\sphinxAtStartPar
Définissez vos réglages pour votre courriel.
Le serveur smpt permettra à \sphinxstyleemphasis{Diacamma Asso} d’envoyer un certain nombre de messages : facture en PDF, mot de passe de connexion, …
Vous pouvez préciser comment réagissent les liens “écrire à tous” réagis avec votre logiciel de messagerie.


\subsection{Les cadres associatifs}
\label{\detokenize{asso/first_step:les-cadres-associatifs}}\begin{quote}

\sphinxAtStartPar
Menu \sphinxstyleemphasis{Général/Nos coordonnées}

\sphinxAtStartPar
Menu \sphinxstyleemphasis{Administration/Modules (conf.)/Configuration des contacts}

\sphinxAtStartPar
Menu \sphinxstyleemphasis{Association/Adhérents/Adhérents cotisants}
\end{quote}

\sphinxAtStartPar
Dans la fenêtre de vos coordonnées, vous pouvez associer des adhérents comme cadres de votre structure.
Utilisez l’outil de recherche et assignez\sphinxhyphen{}leur une fonction.
Vous pouvez également rajouter des fonctions propres à votre structure.

\sphinxAtStartPar
Depuis la fiche de chacun de vos adhérents, vous pouvez donner à des personnes actives un droit de connexion à Diacamma Asso.
Privilégiez une utilisation du logiciel avec un alias et un mot de passe propre à chaque utilisateur. Associez\sphinxhyphen{}leur également les droits correspondant à leurs fonctions au sein de votre structure.
Enfin, évitez autant que possible l’utilisation de l’alias “admin” qui doit être réservé pour des actions de configuration et de maintenance.


\subsection{La gestion documentaire}
\label{\detokenize{asso/first_step:la-gestion-documentaire}}\begin{quote}

\sphinxAtStartPar
Menu \sphinxstyleemphasis{Administration/Modules (conf.)/Dossier}

\sphinxAtStartPar
Menu \sphinxstyleemphasis{Bureautique/Documents/Documents}
\end{quote}

\sphinxAtStartPar
Définissez vos différents dossiers vous permettant d’importer vos documents à classer et à partager.

\sphinxAtStartPar
Après avoir parcouru ces points, votre logiciel \sphinxstyleemphasis{Diacamma Asso} est pleinement opérationnel.
N’hésitez pas à consulter le forum : de nombreuses astuces peuvent vous aider à utiliser au mieux votre logiciel.

\sphinxstepscope


\chapter{Diacamma adhérent}
\label{\detokenize{member/index:diacamma-adherent}}\label{\detokenize{member/index::doc}}
\sphinxAtStartPar
Aide relative aux fonctionnalités de gestion d’adhésion.

\sphinxstepscope


\section{L’adhérent}
\label{\detokenize{member/member:ladherent}}\label{\detokenize{member/member::doc}}
\sphinxAtStartPar
L’adhérent est une personne physique qui a pris une inscription dans votre association.
Consultez la liste de vos adhérents :
\begin{quote}

\sphinxAtStartPar
Menu \sphinxstyleemphasis{Association/Adhérents/Adhérents cotisants}
\end{quote}

\noindent\sphinxincludegraphics{{members}.png}

\sphinxAtStartPar
De cet écran vous pouvez, après avoir filtré vos adhérents sur un ou plusieurs critères, imprimer les listes, les étiquettes ou les cartes d’adhérents.
Sélectionnez et éditez une ligne pour consulter la fiche de l’adhérent.

\sphinxAtStartPar
\sphinxstyleemphasis{Diacamma Asso} sauvegarde plusieurs types d’informations relatives à un adhérent. Vous pouvez les consulter dans la fiche de l’adhérent.

\noindent\sphinxincludegraphics{{file}.png}
\begin{description}
\sphinxlineitem{Cette fiche comporte :}\begin{itemize}
\item {} 
\sphinxAtStartPar
L’identité de l’adhérent : nom, prénom, adresse, téléphone, date et lieu de naissance…

\item {} 
\sphinxAtStartPar
La liste de ses adhésions

\item {} 
\sphinxAtStartPar
D’autres onglets dépendent des extensions installées : liste des diplômes/formations/grades ou situation comptable

\end{itemize}

\end{description}

\sphinxAtStartPar
Si vous avez défini une liste de documents d’inscription pour la saison, vous pouvez, de manière individuelle, cocher les documents rendus. Une impression de l’état de l’ensemble des adhérents en la matière est disponible depuis la liste des adhérents.
Etant donné qu’un adhérent est aussi une personne physique, vous retrouvez ces même fiches depuis la liste des contacts. Il vous est également possible de promouvoir un simple contact physique en adhérent ; vous aurez alors juste à compléter sa fiche d’adhérent.
Enfin, depuis la fiche d’un adhérent, il est possible de lui donner un droit d’accès, ou alias, à \sphinxstyleemphasis{Diacamma}. (voir Les utilisateurs).

\sphinxAtStartPar
En fonction de vos droits de connexion, il vous sera possible de modifier l’identité d’un adhérent, d’ajouter une adhésion passée ou un diplôme obtenu.

\noindent\sphinxincludegraphics{{modify}.png}

\sphinxAtStartPar
\sphinxstylestrong{Attention:} La loi Informatique et liberté n’autorise la conservation de données que si elles sont relatives à l’activité de nos adhérents. Veillez donc à bien adapter vos commentaires à ce cadre législatif et à informer toute personne de son inscription dans votre base de données.
Consultez le texte de loi en cas de doute, et n’hésitez pas à contacter la CNIL en cas de besoin.

\sphinxstepscope


\section{Rechercher un adhérent}
\label{\detokenize{member/member_search:rechercher-un-adherent}}\label{\detokenize{member/member_search::doc}}\begin{quote}

\sphinxAtStartPar
Menu \sphinxstyleemphasis{Association/Adhérents/Recherche d’adhérents}
\end{quote}

\noindent\sphinxincludegraphics{{search}.png}

\sphinxAtStartPar
Vous pouvez alors faire une recherche suivant des critères variés portant sur l’identité ou les adhésions.
Vous pouvez également rechercher des adhérents sur des “documents demandés” ou sur un champ personnalisé.

\sphinxAtStartPar
\sphinxstylestrong{Remarque:} si vous faites une recherche avec des critères liés à une cotisation (équipe, documents, numéro licence, …) n’oubliez pas de préciser la saison de recherche. Le résultat donne une liste d’adhérents correspondant aux critères fournis.

\sphinxAtStartPar
Il est possible de fusionner plusieurs fiches d’une même personne.
Pour cela vous devez préciser la personne principale, l’outil supprimera les autres fiches après avoir déplacé toutes leurs références sur l’enregistrement principal.
Si vous voulez supprimer un adhérent, celui\sphinxhyphen{}ci ne devra pas avoir eu d’activité.

\sphinxAtStartPar
Depuis l’outil de recherche, vous pouvez aussi rechercher les fiches adhérents doublons. Cela vous permet d’afficher la liste des adhérents ayant les nom et prénom similaires.
De là, vous pouvez également fusionner ou supprimer la ou les fiches redondantes.

\sphinxstepscope


\section{Les cotisations}
\label{\detokenize{member/subscription:les-cotisations}}\label{\detokenize{member/subscription::doc}}
\sphinxAtStartPar
Ce qui caractèrise un adhérent par rapport à un contact physique, c’est le fait qu’il soit membre de votre association et qu’il soit à jour de cotisation.

\sphinxAtStartPar
Une cotisation symbolise l’acte d’adhésion à votre structure pendant un laps de temps défini.
Une personne sera donc considérée comme adhérente de votre association à une date donnée si elle a souscrit à une cotisation sur une période de temps comprenant cette date.


\subsection{Les états d’une cotisation}
\label{\detokenize{member/subscription:les-etats-d-une-cotisation}}
\sphinxAtStartPar
La cotisation d’un adhérent peut avoir plusieurs états :
\begin{itemize}
\item {} 
\sphinxAtStartPar
en attente :

\end{itemize}

\sphinxAtStartPar
État transitoire accessible seulement si vous souhaitez donner à vos adhérents la possibilité de demander en ligne une cotisation.
Cet état nécessite alors une modération pour la passer à \sphinxstyleemphasis{en création}.
\begin{itemize}
\item {} 
\sphinxAtStartPar
en création :

\end{itemize}

\sphinxAtStartPar
État correspondant à une préinscription.
La cotisation est alors associée à un devis pour approbation de la part de l’adhérent.
En cas de non acceptation, cette cotisation est \sphinxstyleemphasis{annulée} ou simplement supprimée.
\begin{itemize}
\item {} 
\sphinxAtStartPar
validée :

\end{itemize}

\sphinxAtStartPar
État général correspondant à un adhérent à part entière dans votre association.
Le devis est alors transformé en facture.
La cotisation est associée à une facture.
\begin{itemize}
\item {} 
\sphinxAtStartPar
annulée :

\end{itemize}

\sphinxAtStartPar
État permettant de suivre les anciens adhérants n’ayant pas souhaité confirmer leur adhésion.
\begin{itemize}
\item {} 
\sphinxAtStartPar
radiée :

\end{itemize}

\sphinxAtStartPar
État permettant de noter qu’un adhérent a été exclu de l’association.
À noter qu’en cas d’erreur, il est possible de rééditer cette cotisation pour la repasser à \sphinxstyleemphasis{en création}. Un nouveau devis de cotisation est alors régénéré.

\sphinxstepscope


\section{Renouvellements}
\label{\detokenize{member/renew:renouvellements}}\label{\detokenize{member/renew::doc}}
\sphinxAtStartPar
Quand vous commencez une nouvelle saison, vous avez un certain nombre d’adhérents qui renouvellent leur adhésion. Si vous avez déjà enregistré cette personne dans votre logiciel, vous pouvez rechercher sa fiche et lui rajouter une licence pour la nouvelle saison.

\sphinxAtStartPar
Si vous avez beaucoup de renouvellements ce traitement serait très long !
\begin{quote}

\sphinxAtStartPar
Menu \sphinxstyleemphasis{Association/Adhérents/Adhérents à renouveler}
\end{quote}

\sphinxAtStartPar
Dans cette liste, vous retrouvez tous les adhérents de la saison précédente qui n’ont pas encore été renouvelés.

\sphinxAtStartPar
Pour leur ajouter la nouvelle cotisation, deux solutions :
\begin{itemize}
\item {} \begin{description}
\sphinxlineitem{De manière unitaire :}
\sphinxAtStartPar
ouvrez chaque fiche et ajoutez la nouvelle licence

\end{description}

\item {} \begin{description}
\sphinxlineitem{De manière globale :}
\sphinxAtStartPar
sélectionnez plusieurs fiches dans la liste et cliquez sur « Renouveler ». Chaque personne sera alors renouvelée dans les mêmes conditions que la saison précédente (même cotisation, même N° de licence, même équipe, même activité)

\end{description}

\end{itemize}

\sphinxstepscope


\section{Prestations de cotisation}
\label{\detokenize{member/prestation:prestations-de-cotisation}}\label{\detokenize{member/prestation::doc}}
\sphinxAtStartPar
Cela permet d’associer une équipe/cours à un article facturable afin de proposer un choix de prestations supplémentaires au moment de la saisie de la cotisation.

\sphinxAtStartPar
Vous choisissez vos prestations au moment de votre prise de cotisation ou via un écran de gestion spécifique.
Automatiquement, votre adhérent est alors associé à la bonne catégorie d’équipes/cours définie par la prestation.
De plus, dans la facture d’adhésion est ajouté alors l’article relatif à cette prestation.

\sphinxAtStartPar
Pour activer ce mode, vous devez configurer dans le menu \sphinxstyleemphasis{Administration/Modules (conf.)/Catégories}, le paramètre \sphinxstyleemphasis{Activer les équipes} = « Avec prestation »
Une fois alors votre Diacamma rafraichi, un nouvelle écran \sphinxstyleemphasis{Association/Liste de prestations}
\begin{quote}

\noindent\sphinxincludegraphics{{listprestation}.png}
\end{quote}


\subsection{Créer un prestation}
\label{\detokenize{member/prestation:creer-un-prestation}}
\sphinxAtStartPar
Depuis l’écran \sphinxstyleemphasis{Association/Liste de prestations}, il vous est possible de créer une nouvelle prestation.

\sphinxAtStartPar
Pour cela, cliquez sur le bouton « Créer ».
\begin{quote}

\noindent\sphinxincludegraphics{{newprestation}.png}
\end{quote}

\sphinxAtStartPar
Vous pouvez alors créer une prestation soit en créant une équipe/cours, soit en l’associant à une existante.
Si vous gerez des activités, vous devez alors associer cette prestation à une activité existante.
De plus, vous devez également définir un article de facturation à cette prestation.

\sphinxAtStartPar
Vous pouvez bien entendu, ensuite modifier cette prestation: nom et description de son équipe/cours, son activité et son article.

\sphinxAtStartPar
Si vous le voulez alors, vous pouvez également lui associé d’autres articles afin que celle\sphinxhyphen{}ci comporte plusieurs tarif.
Cochez « Utiliser le multi\sphinxhyphen{}prix » pour que l’interface vous propose l’ajout, la modification ou la suppression d’un article de prix.
\begin{quote}

\noindent\sphinxincludegraphics{{modifprestation}.png}
\end{quote}

\sphinxAtStartPar
Si vous souhaitez supprimer une prestation, il vous sera demandé ce que vous souhaitez faire de l’équipe/cours associée: la désactiver, la supprimer ou la laisser.


\subsection{Associer des pratiquants}
\label{\detokenize{member/prestation:associer-des-pratiquants}}
\sphinxAtStartPar
A chacune de ces prestations, vous pouvez bien sur ajouter des pratiquants.
\begin{quote}

\noindent\sphinxincludegraphics{{showprestation}.png}
\end{quote}

\sphinxAtStartPar
Via le bouton « Ajouter », rechercher une (ou plusieurs) fiche adhérent qui sera alors associé à cette prestation.
Vous pouvez de là également créer une nouvelle fiche.
Si cet adhérent n’est pas encore cotisant à votre structure, il vous sera demandé un type de cotisation.
Si la cotisation a plusieurs prix possible, il vous sera demandé quelle tarification utiliser.

\sphinxAtStartPar
Dans le cas où un adhérent a sa cotisation « en création », les articles de la prestations choisis serons alors automatiquement ajouter au devis associé à cette cotisation.
Si sa cotisation est « validée », une facture est alors généré afin de prendre en compte cette nouvelle prestation.
Dans le cas où l’adhérent est retiré d’une prestation, un avoir équivalent est alors créé.


\subsection{Gérer des prestations}
\label{\detokenize{member/prestation:gerer-des-prestations}}
\sphinxAtStartPar
Bien entendu, vos prestations ne sont pas figées.

\sphinxAtStartPar
Vous pouvez vouloir inverser des pratiquants, d’une prestation à une autre.
Pour cela, cliquez sur « Permuter » après avoir sélectionné 2 prestations: un écran vous invitera à associer les bons pratiquants au bon groupe.
\begin{quote}

\noindent\sphinxincludegraphics{{swapingprestation}.png}
\end{quote}

\sphinxAtStartPar
Dans la même idée, le bouton « Dédoubler » vous permet de créer une nouvelle prestation d’après une ancienne et d’ensuite pouvoir permuter les pratiquants comme vous le souhaitez.

\sphinxAtStartPar
Le bouton « Fusion » permet, quand à lui, de fusionner en une seule prestation celles que vous aurez sélectionnées.
Les équipes/cours non conservées sont alors supprimées.

\sphinxAtStartPar
Avec cette gestion, la facturation s’en retrouvera alors automatiquement impacté avec la création de devis, facture ou avoir, suivant les modifications apportées.

\sphinxstepscope


\section{Statistiques des adhérents}
\label{\detokenize{member/statistic:statistiques-des-adherents}}\label{\detokenize{member/statistic::doc}}\begin{quote}

\sphinxAtStartPar
Menu \sphinxstyleemphasis{Association/Adhérents/Statistiques}
\end{quote}

\noindent\sphinxincludegraphics{{statistiques}.png}

\sphinxAtStartPar
Vous avez la possibilité d’éditer les statistiques d’adhésion d’une saison donnée.

\sphinxAtStartPar
Ce document vous donne les effectifs par ville et par genre. Il est possible aussi de distinguer les différentes durées de cotisations.

\sphinxstepscope


\section{Reçus fiscaux}
\label{\detokenize{member/taxreceipt:recus-fiscaux}}\label{\detokenize{member/taxreceipt::doc}}
\sphinxAtStartPar
Si votre association est d’intérêt général, elle peut délivrer des reçus fiscaux à ses donateurs et à ses membres. Vous pouvez utiliser \sphinxstyleemphasis{Diacamma} pour cela.


\subsection{Configuration}
\label{\detokenize{member/taxreceipt:configuration}}
\sphinxAtStartPar
\sphinxstylestrong{Indiquer les codes comptables des produits pouvant donner droit à une déduction fiscale}
\begin{quote}

\sphinxAtStartPar
Menu \sphinxstyleemphasis{Administration/Configuration générale} \sphinxhyphen{} onglet « Adhérents »
\end{quote}

\sphinxAtStartPar
Mettez à jour le champ « code comptable des reçus fiscaux » avec les comptes de produits concernés.
Après rafraîchissement de l’application, le menu \sphinxstyleemphasis{Comptabilité/Reçus fiscaux} est maintenant disponible.

\sphinxAtStartPar
\sphinxstylestrong{Ajouter une image de signature}
\begin{quote}

\sphinxAtStartPar
Menu \sphinxstyleemphasis{Bureautique/Gestion de fichiers et de documents/Documents}
\end{quote}

\sphinxAtStartPar
Ajoutez dans le \sphinxstyleemphasis{Gestionnaire de documents} le fichier image devant être utilisé comme cachet ou signature de votre association.
\begin{quote}

\sphinxAtStartPar
Menu \sphinxstyleemphasis{Administration/Configuration générale} \sphinxhyphen{} onglet « Gestion documentaire »
\end{quote}

\sphinxAtStartPar
Dans un second temps, mettez à jour le champ « image de cachet ou de signature ».
Cette image, intégrée dans le modèle d’impression par défaut, sera insérée automatiquement dans le reçu au format PDF, zone « signature ».

\sphinxAtStartPar
\sphinxstylestrong{Préciser la justification du reçu fiscal}

\sphinxAtStartPar
L’association doit indiquer sur les reçus à quel titre elle les émet ce reçu.
En général, elle précise la date de parution au Journal Officiel de sa situation d’utilité publique.
\begin{quote}

\sphinxAtStartPar
Menu \sphinxstyleemphasis{Général/Nos coordonnées}
\end{quote}

\sphinxAtStartPar
Éditez la fiche de votre association et renseignez le champ « N° SIRET/SIREN ». Ce champ est ajouté par défaut en pied de page du reçu.

\sphinxAtStartPar
\sphinxstylestrong{Ajouter le logo de votre structure}

\sphinxAtStartPar
Afin d’éviter d’avoir le logo par défaut dans vos reçus, enregistrez sous \sphinxstyleemphasis{Diacamma} le logo de votre association.
\begin{quote}

\sphinxAtStartPar
Menu \sphinxstyleemphasis{Général/Nos coordonnées}
\end{quote}

\sphinxAtStartPar
Éditez la fiche de votre association et associez\sphinxhyphen{}lui une image.


\subsection{Génération des reçus}
\label{\detokenize{member/taxreceipt:generation-des-recus}}\begin{quote}

\sphinxAtStartPar
Menu \sphinxstyleemphasis{Comptabilité/Reçus fiscaux}
\end{quote}

\sphinxAtStartPar
Visualisez les reçus fiscaux déjà produits et filtrez\sphinxhyphen{}les sur l’année civile.
L’ouverture du droit à déduction fiscale nait du règlement du produit ouvrant droit à déduction. C’est pour cela que le reçu fiscal est établi sur l’année du règlement. En cas de règlement fractionné, c’est le dernier versement qui fait naître le droit à déduction.
\begin{quote}

\noindent\sphinxincludegraphics{{taxreceipts}.png}
\end{quote}

\sphinxAtStartPar
Les reçus fiscaux de l’année sélectionnée peuvent être générés, via le bouton « Contrôle » (situé en bas à droite) :
\begin{itemize}
\item {} 
\sphinxAtStartPar
L’outil extrait tous les mouvements satisfaisant aux conditions suivantes :
\begin{itemize}
\item {} 
\sphinxAtStartPar
les codes comptables doivent correspondre à des produits ouvrant droit à déduction fiscale (Menu \sphinxstyleemphasis{/Configuration générale})

\item {} 
\sphinxAtStartPar
les écritures associées doivent avoir été validées

\item {} 
\sphinxAtStartPar
les créances liées aux mouvements doivent être lettrées afin d’attester qu’elles sont bien réglées

\item {} 
\sphinxAtStartPar
le règlement doit avoir eu lieu sur l’année spécifiée

\item {} 
\sphinxAtStartPar
les mouvements ne doivent pas être déjà associés à des reçus fiscaux

\end{itemize}

\end{itemize}

\sphinxAtStartPar
Les mouvements satisfaisant à ces cinq conditions sont reportés dans le reçu de chaque tiers concerné.
\begin{itemize}
\item {} 
\sphinxAtStartPar
Attribution d’un numéro unique (spécifique à l’année civile) à tout reçu fiscal

\item {} \begin{description}
\sphinxlineitem{Possibilité de gérer des reçus fiscaux pour « abandon de frais »}
\sphinxAtStartPar
Ces dons peuvent prendre plusieurs formes (argent, abandon de revenus ou de produits, renonciation expresse à des frais engagés dans le cadre d’une activité bénévole respectant certaines conditions)

\end{description}

\end{itemize}

\sphinxAtStartPar
Une fois généré, chaque reçu pourra être imprimé en PDF et pourra être envoyé par courriel (tout comme les factures).

\sphinxAtStartPar
\sphinxstylestrong{Attention :} la génération de reçus fiscaux est définitive. Comme ils correspondent à la réalité d’une comptabilité validée, il n’est pas possible de les corriger ou de les annuler.

\sphinxAtStartPar
\sphinxstylestrong{Note:} Une version PDF des reçus est automatiquement sauvegardée dans le \sphinxstyleemphasis{Gestionnaire de documents}, dans un répertoire ayant pour nom l’année civile. Si ce répertoire n’existe pas, il est créé.

\sphinxstepscope


\section{Configuration}
\label{\detokenize{member/config:configuration}}\label{\detokenize{member/config::doc}}
\sphinxAtStartPar
Pour vous permettre d’exploiter au mieux le logiciel, certains paramétrages sont nécessaires.


\subsection{Générale}
\label{\detokenize{member/config:generale}}\begin{quote}

\sphinxAtStartPar
Menu \sphinxstyleemphasis{Administration/Configuration générale}, onglet \sphinxstyleemphasis{Adhérents}
\end{quote}

\sphinxAtStartPar
Vous pouvez modifier les paramètrages généraux relatifs à la gestion des adhésions.
\begin{quote}

\noindent\sphinxincludegraphics{{conf_general}.png}
\end{quote}


\subsection{Saisons et cotisations}
\label{\detokenize{member/config:saisons-et-cotisations}}\begin{quote}

\sphinxAtStartPar
Menu \sphinxstyleemphasis{Administration/Modules (conf.)/Les saisons et les cotisations}
\end{quote}

\sphinxAtStartPar
Vous pouvez modifier les paramètres de la gestion des licences sportives, en l’occurrence, les saisons sportives et les types de cotisations.

\sphinxAtStartPar
\sphinxstylestrong{Les Saisons}
\begin{quote}

\noindent\sphinxincludegraphics{{season_list}.png}
\end{quote}

\sphinxAtStartPar
Ici, vous pourrez ajouter de nouvelles saisons au fur et à mesure de l’utilisation du logiciel. Vous pourrez également déterminer la saison courante (dite active).
De plus, chaque saison est découpée en quatre périodes et en douze mois.

\sphinxAtStartPar
C’est la plus petite date de début et la plus grande de fin qui définissent la plage de votre saison. Vos saisons peuvent être d’une durée supérieure à un an et peuvent se chevaucher.
Vous pouvez modifier les dates de début et de fin de chaque période. Vous pouvez aussi ajouter ou supprimer une période, mais vous devez toujours en avoir au moins deux.
Deux périodes peuvent se chevaucher ou être disjointes (dans ce cas, un message d’avertissement vous prévient).
Le premier des douze mois commence le mois de la plus petite date de début de période. Même si votre saison couvre plus d’une année calendaire, il n’y aura pas de treizième mois dans votre saison.

\sphinxAtStartPar
Vous pouvez associer à chaque saison une liste de documents que chaque adhérent devra vous fournir pour finaliser son inscription.
\begin{quote}

\noindent\sphinxincludegraphics{{documents}.png}
\end{quote}

\sphinxAtStartPar
\sphinxstylestrong{Les types de cotisation}
\begin{quote}

\noindent\sphinxincludegraphics{{cotisations}.png}
\end{quote}

\sphinxAtStartPar
Ici vous pourrez saisir les différents types de cotisation proposés par votre association. Par exemple, pour une association pratiquant plusieurs activités sportives distinctes, vous pouvez avoir un type de cotisation pour chaque activité, un autre pour plusieurs de ces activités, et encore des types différents selon une pratique en compétition ou hors compétition des activités.

\sphinxAtStartPar
Quatre modes de durées différentes peuvent être affectées à un type de cotisation :
\begin{itemize}
\item {} 
\sphinxAtStartPar
Annuelle : cotisation couvrant l’ensemble de la saison.

\item {} 
\sphinxAtStartPar
Périodique : cotisation couvrant une période (4 par défaut) de la saison.

\item {} 
\sphinxAtStartPar
Mensuelle : cotisation couvrant un des douze mois de la saison.

\item {} 
\sphinxAtStartPar
Calendaire : cotisation couvrant une année calendaire. Cette cotisation peut donc être à cheval sur deux saisons.

\end{itemize}

\sphinxAtStartPar
Pour le lien avec le module \sphinxstyleemphasis{Facturier} vous devez définir un prix de vente de votre cotisation en créant et en associant des articles.
Vous pouvez rattacher plusieurs articles à  une cotisation. Cela vous permet de distinguer, par exemple, la part de cotisation relative à votre club, de la licence de votre fédération.
Avec ces liens entre les cotisations et les articles,vous pourrez générer automatiquement des factures lors de vos procédures d’adhésion. Si une cotisation n’est liée au aucun article, aucune facture ne sera émise.

\sphinxAtStartPar
De même, vous pouvez aussi personnaliser le code comptable du tiers associé à vos adhérants dans le cas d’un création automatique.


\subsection{Catégories}
\label{\detokenize{member/config:categories}}
\sphinxAtStartPar
Le menu \sphinxstyleemphasis{Administration/Modules (conf.)/Catégories} vous permet de modifier ce qui peut catégoriser un adhérent : les catégories d’âge, les équipes ou cours et les activités ainsi que la possibilité d’activer ou non ces différentes classifications.

\sphinxAtStartPar
Vous pouvez ne pas vouloir utiliser certaines catégories. Pour cela, désactivez\sphinxhyphen{}les depuis l’écran de paramétrage.
De la même façon, vous pouvez préciser si vous souhaitez pouvoir créer automatiquement une connexion par adhérent actif, afficher un numéro d’adhérent ou gérer des numéros de licence.
Vous pouvez également personnaliser la désignation “équipe” et “activité”.
\begin{quote}

\noindent\sphinxincludegraphics{{categories}.png}
\end{quote}

\sphinxAtStartPar
\sphinxstylestrong{Les âges}

\sphinxAtStartPar
Vous pourrez ici renseigner les catégories d’âges existantes dans votre association avec un nom de catégorie, une année (de naissance) de début et de fin de la catégorie.

\sphinxAtStartPar
Vous n’aurez pas besoin de changer les valeurs des années de naissance ultérieurement : le décalage est effectué automatiquement d’année en année.
\begin{quote}

\noindent\sphinxincludegraphics{{age}.png}
\end{quote}

\sphinxAtStartPar
\sphinxstylestrong{Les équipes/cours}

\sphinxAtStartPar
Vous gérez différentes équipes ou différents cours et vous souhaitez pouvoir gérer vos adhérents selon ce critère.
Renseignez\sphinxhyphen{}les ici, vous pourrez alors affecter des adhérents à ces équipes ou cours et ainsi les retrouver plus facilement.
\begin{quote}

\noindent\sphinxincludegraphics{{team}.png}
\end{quote}

\sphinxAtStartPar
\sphinxstylestrong{Les activités}

\sphinxAtStartPar
Vous gérez différentes activités (par exemple plusieurs arts martiaux) dans votre association ? Les renseigner ici vous permettra ensuite de classer vos adhérents en fonction de ces différentes activités, mais aussi de saisir pour eux plusieurs licences par an si nécessaire.

\sphinxAtStartPar
Exemple : une association regroupant judo et karaté, et donc affiliée à deux fédérations sportives différentes.
Vous pourriez alors saisir 2 licences par adhérent (sous réserve que vos adhérents pratiquent les deux sports et soient licenciés des deux fédérations).
\begin{quote}

\noindent\sphinxincludegraphics{{activity}.png}
\end{quote}

\sphinxstepscope


\chapter{Diacamma évenement}
\label{\detokenize{event/index:diacamma-evenement}}\label{\detokenize{event/index::doc}}
\sphinxAtStartPar
Aide relative aux fonctionnalités de gestion d’évenement.

\sphinxstepscope


\section{Création d’un évenement}
\label{\detokenize{event/newevent:creation-d-un-evenement}}\label{\detokenize{event/newevent::doc}}
\sphinxAtStartPar
Un évenement, pour \sphinxstyleemphasis{Diacamma}, correspond à la gestion d’une activité particulière de votre structure.
\begin{description}
\sphinxlineitem{Elle peut être de deux types:}\begin{itemize}
\item {} 
\sphinxAtStartPar
Un passage d’examen
Pour vous aider à gérer un examen au sein de votre association afin de valider pour vos pratiquants un grade, un diplôme ou un niveau.
Il est nécessaire pour cela de configurer la liste des grades/niveaux relatifs à votre structure.

\item {} 
\sphinxAtStartPar
Un stage ou une sortie
Ce type d’évenement va correspond à un grand nombre de cas d’activité dans lequel vous souhaitez gérer une inscription.

\end{itemize}

\end{description}

\sphinxAtStartPar
La création d’un évenement est assez simple.
Commencez simplement à préciser le type, la description et les dates de celui\sphinxhyphen{}ci.

\sphinxAtStartPar
Assignez alors à cet évenement un équipe d’organisation (dont un responsable) ainsi que des participants.
Les participants peuvent aussi bien être adhérent ou de simple contact: une colonne permet rapidement de connaitre leur statut.

\sphinxAtStartPar
A chaque participant, vous pouvez assigner un article de facture (un article par défaut simplifie également cette gestion).
Une fois l’évenement validé, une facture est ainsi générée en lien avec l’article du participant.

\sphinxAtStartPar
Dans le cas d’un passage d’examen, à la validation, vous devez également préciser le résultat de chacun des participants.
Tout les participants étant un adhérent de la structure auront une modification de leur fiche afin de faire figurer, le cas échant, leur nouveau grade, diplôme ou niveau.

\sphinxstepscope


\section{Statistique de formation}
\label{\detokenize{event/statistic:statistique-de-formation}}\label{\detokenize{event/statistic::doc}}
\sphinxAtStartPar
Un outil statistique simple vous permet de d’avoir rapidement un résumé pour une saison donnée du nombre de diplômes attribué pour cette periode de temps.

\sphinxstepscope


\section{Configurations}
\label{\detokenize{event/config:configurations}}\label{\detokenize{event/config::doc}}
\sphinxAtStartPar
La configuration du gestionnaire d’évenement est seulement relative à la configuration des grades, diplômes ou niveaux.

\sphinxAtStartPar
Dans le menu \sphinxstyleemphasis{Administration/Configuration des diplômes} vous pouvez préciser la dénomination de vos diplômes
ainsi que saisir hierachiquement les diplômes et éventuellement les sous\sphinxhyphen{}diplômes relatifs à votre structures.



\renewcommand{\indexname}{Index}
\printindex
\end{document}